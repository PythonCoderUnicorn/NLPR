% Options for packages loaded elsewhere
\PassOptionsToPackage{unicode}{hyperref}
\PassOptionsToPackage{hyphens}{url}
\PassOptionsToPackage{dvipsnames,svgnames,x11names}{xcolor}
%
\documentclass[
  letterpaper,
  DIV=11,
  numbers=noendperiod]{scrreprt}

\usepackage{amsmath,amssymb}
\usepackage{lmodern}
\usepackage{iftex}
\ifPDFTeX
  \usepackage[T1]{fontenc}
  \usepackage[utf8]{inputenc}
  \usepackage{textcomp} % provide euro and other symbols
\else % if luatex or xetex
  \usepackage{unicode-math}
  \defaultfontfeatures{Scale=MatchLowercase}
  \defaultfontfeatures[\rmfamily]{Ligatures=TeX,Scale=1}
\fi
% Use upquote if available, for straight quotes in verbatim environments
\IfFileExists{upquote.sty}{\usepackage{upquote}}{}
\IfFileExists{microtype.sty}{% use microtype if available
  \usepackage[]{microtype}
  \UseMicrotypeSet[protrusion]{basicmath} % disable protrusion for tt fonts
}{}
\makeatletter
\@ifundefined{KOMAClassName}{% if non-KOMA class
  \IfFileExists{parskip.sty}{%
    \usepackage{parskip}
  }{% else
    \setlength{\parindent}{0pt}
    \setlength{\parskip}{6pt plus 2pt minus 1pt}}
}{% if KOMA class
  \KOMAoptions{parskip=half}}
\makeatother
\usepackage{xcolor}
\setlength{\emergencystretch}{3em} % prevent overfull lines
\setcounter{secnumdepth}{5}
% Make \paragraph and \subparagraph free-standing
\ifx\paragraph\undefined\else
  \let\oldparagraph\paragraph
  \renewcommand{\paragraph}[1]{\oldparagraph{#1}\mbox{}}
\fi
\ifx\subparagraph\undefined\else
  \let\oldsubparagraph\subparagraph
  \renewcommand{\subparagraph}[1]{\oldsubparagraph{#1}\mbox{}}
\fi


\providecommand{\tightlist}{%
  \setlength{\itemsep}{0pt}\setlength{\parskip}{0pt}}\usepackage{longtable,booktabs,array}
\usepackage{calc} % for calculating minipage widths
% Correct order of tables after \paragraph or \subparagraph
\usepackage{etoolbox}
\makeatletter
\patchcmd\longtable{\par}{\if@noskipsec\mbox{}\fi\par}{}{}
\makeatother
% Allow footnotes in longtable head/foot
\IfFileExists{footnotehyper.sty}{\usepackage{footnotehyper}}{\usepackage{footnote}}
\makesavenoteenv{longtable}
\usepackage{graphicx}
\makeatletter
\def\maxwidth{\ifdim\Gin@nat@width>\linewidth\linewidth\else\Gin@nat@width\fi}
\def\maxheight{\ifdim\Gin@nat@height>\textheight\textheight\else\Gin@nat@height\fi}
\makeatother
% Scale images if necessary, so that they will not overflow the page
% margins by default, and it is still possible to overwrite the defaults
% using explicit options in \includegraphics[width, height, ...]{}
\setkeys{Gin}{width=\maxwidth,height=\maxheight,keepaspectratio}
% Set default figure placement to htbp
\makeatletter
\def\fps@figure{htbp}
\makeatother
\newlength{\cslhangindent}
\setlength{\cslhangindent}{1.5em}
\newlength{\csllabelwidth}
\setlength{\csllabelwidth}{3em}
\newlength{\cslentryspacingunit} % times entry-spacing
\setlength{\cslentryspacingunit}{\parskip}
\newenvironment{CSLReferences}[2] % #1 hanging-ident, #2 entry spacing
 {% don't indent paragraphs
  \setlength{\parindent}{0pt}
  % turn on hanging indent if param 1 is 1
  \ifodd #1
  \let\oldpar\par
  \def\par{\hangindent=\cslhangindent\oldpar}
  \fi
  % set entry spacing
  \setlength{\parskip}{#2\cslentryspacingunit}
 }%
 {}
\usepackage{calc}
\newcommand{\CSLBlock}[1]{#1\hfill\break}
\newcommand{\CSLLeftMargin}[1]{\parbox[t]{\csllabelwidth}{#1}}
\newcommand{\CSLRightInline}[1]{\parbox[t]{\linewidth - \csllabelwidth}{#1}\break}
\newcommand{\CSLIndent}[1]{\hspace{\cslhangindent}#1}

\KOMAoption{captions}{tableheading}
\makeatletter
\makeatother
\makeatletter
\@ifpackageloaded{bookmark}{}{\usepackage{bookmark}}
\makeatother
\makeatletter
\@ifpackageloaded{caption}{}{\usepackage{caption}}
\AtBeginDocument{%
\ifdefined\contentsname
  \renewcommand*\contentsname{Table of contents}
\else
  \newcommand\contentsname{Table of contents}
\fi
\ifdefined\listfigurename
  \renewcommand*\listfigurename{List of Figures}
\else
  \newcommand\listfigurename{List of Figures}
\fi
\ifdefined\listtablename
  \renewcommand*\listtablename{List of Tables}
\else
  \newcommand\listtablename{List of Tables}
\fi
\ifdefined\figurename
  \renewcommand*\figurename{Figure}
\else
  \newcommand\figurename{Figure}
\fi
\ifdefined\tablename
  \renewcommand*\tablename{Table}
\else
  \newcommand\tablename{Table}
\fi
}
\@ifpackageloaded{float}{}{\usepackage{float}}
\floatstyle{ruled}
\@ifundefined{c@chapter}{\newfloat{codelisting}{h}{lop}}{\newfloat{codelisting}{h}{lop}[chapter]}
\floatname{codelisting}{Listing}
\newcommand*\listoflistings{\listof{codelisting}{List of Listings}}
\makeatother
\makeatletter
\@ifpackageloaded{caption}{}{\usepackage{caption}}
\@ifpackageloaded{subcaption}{}{\usepackage{subcaption}}
\makeatother
\makeatletter
\@ifpackageloaded{tcolorbox}{}{\usepackage[many]{tcolorbox}}
\makeatother
\makeatletter
\@ifundefined{shadecolor}{\definecolor{shadecolor}{rgb}{.97, .97, .97}}
\makeatother
\makeatletter
\makeatother
\ifLuaTeX
  \usepackage{selnolig}  % disable illegal ligatures
\fi
\IfFileExists{bookmark.sty}{\usepackage{bookmark}}{\usepackage{hyperref}}
\IfFileExists{xurl.sty}{\usepackage{xurl}}{} % add URL line breaks if available
\urlstyle{same} % disable monospaced font for URLs
\hypersetup{
  pdftitle={NLPR},
  pdfauthor={Zane Dax (She/They)},
  colorlinks=true,
  linkcolor={blue},
  filecolor={Maroon},
  citecolor={Blue},
  urlcolor={Blue},
  pdfcreator={LaTeX via pandoc}}

\title{NLPR}
\author{Zane Dax (She/They)}
\date{2022-09-12}

\begin{document}
\maketitle
\ifdefined\Shaded\renewenvironment{Shaded}{\begin{tcolorbox}[boxrule=0pt, breakable, borderline west={3pt}{0pt}{shadecolor}, enhanced, frame hidden, interior hidden, sharp corners]}{\end{tcolorbox}}\fi

\renewcommand*\contentsname{Table of contents}
{
\hypersetup{linkcolor=}
\setcounter{tocdepth}{2}
\tableofcontents
}
\bookmarksetup{startatroot}

\hypertarget{welcome-to-nlpr}{%
\chapter*{Welcome to NLPR}\label{welcome-to-nlpr}}
\addcontentsline{toc}{chapter}{Welcome to NLPR}

This is an online book for \emph{Natural Language Processing in R}.

This work is by Zane Dax and currently has no license.

\bookmarksetup{startatroot}

\hypertarget{preface}{%
\chapter*{Preface}\label{preface}}
\addcontentsline{toc}{chapter}{Preface}

This book was born out of my personal need to have a reference to
\textbf{text analysis} that was light on the text and more on the code.
Having been taught the basics of text analysis from various RLadies
meetups who reference \emph{Text Mining with R} by Julia Silge and David
Robinson. This book is to be a reference for those who already read
\emph{Text Mining with R} and to see what the \textbf{Quanteda} library
offers while keeping the tutorials together.

There are numerous YouTube videos covering the TidyText library with
\texttt{gutenbergr} or \texttt{janeaustenr} libraries to show the
functionality of the tidytext library. The Quanteda library has some
videos and this book goes through their tutorial. This book aims to be
an inclusive reference for text analysis, as to help as many people as
possible.

As I learn more skills and use-cases this book will expand, where I will
include tutorials from YouTube or elsewhere as to be a quick guide. In
some sense a refresher on what functions to call for a particular task.
The goal is to show as many real world examples as I come across and/or
have access to.

\hypertarget{outline}{%
\section*{Outline}\label{outline}}
\addcontentsline{toc}{section}{Outline}

\begin{itemize}
\item
  Chapter 1 Quanteda - goes through the official Quanteda tutorial but
  different examples and concise corpus dataframes in examples. This
  tutorial of the library is simplified with verbose variable names and
  code chunk comments.
\item
  Chapter 2 TidyText - goes through most of the chapters and topics of
  the book but without much depth or explanation. The data used in this
  section uses data from Twitter and from some webscraping as to
  differentiate this book and Julia Silge and David Robinson's work.
  Since this section uses different data and far less text for
  explanations, it should be fair that for those who need further
  explanation, \emph{please read the source material}.
\end{itemize}

\hypertarget{future-plans}{%
\section*{Future plans}\label{future-plans}}
\addcontentsline{toc}{section}{Future plans}

This book has plans on expanding libraries and examples from various
sources. Part of the plan is to expand part of this book to Python Text
Analysis using the SpaCy library.

\hypertarget{about-this-book}{%
\section*{About this book}\label{about-this-book}}
\addcontentsline{toc}{section}{About this book}

This book has no math formulas, as this book is for those who just want
to code and/or refresh their knowledge.

This book uses the \texttt{\%\textgreater{}\%} magrittr pipe and uses
\texttt{=} instead of the \texttt{\textless{}-} assignments. This book
tries to use the full library and function syntax such as
\texttt{tidytext::unnest\_tokens()} when the function being called is
not widely known or used.

Code comments were used in effort to inform the reader of what is
happening, what some of the output means and the section queries. This
book aims on being clear and to the point, where the reader
\emph{should} be able to go to any section and follow along.

The data visualizations all use the \texttt{ggdark::darkmode()}
function, as to blend with the book dark theme, however this is optional
for the reader and can omit this line of code.

\bookmarksetup{startatroot}

\hypertarget{quanteda}{%
\chapter{Quanteda}\label{quanteda}}

The content in the section has most of the content from Quanteda's
tutorials, with each section adapted to be similar but different. Some
topics are not covered in this section such as Wordfish, Regular
regression classifier, Topic Models, etc. as to not reproduce a full
tutorial series but to show main parts of the library.

This section is for the Quanteda tutorials, which has 3 components
(object types).

The 3 object types:

\begin{enumerate}
\def\labelenumi{\arabic{enumi}.}
\item
  \texttt{corpus} = character strings and variables in data frames,
  combines texts with document level variables
\item
  \texttt{tokens} = tokens in a list of vectors, keeping the position of
  words
\item
  \texttt{document-feature-matrix} (DFM) = represents frequencies in a
  document in a matrix, no positions of words, can use bag-of-words
  analysis
\end{enumerate}

\bookmarksetup{startatroot}

\hypertarget{tidytext}{%
\chapter{TidyText}\label{tidytext}}

This section is about the \emph{Text Mining with R} book by Julia Silge
and David Robinson, but delivered in a different way and with different
examples. This section is not trying to copy nor reproduce their work,
but with the goal of showing real world examples. This section will not
do much explaining of concepts as the original material covers that.

If you read this book without reading the \emph{Text Mining with R} book
first, you will still hopefully be able to follow along and understand
the functions. This section covers most of the book but not everything,
the code here is reduced and simplified.

\bookmarksetup{startatroot}

\hypertarget{summary}{%
\chapter{Summary}\label{summary}}

In the Quanteda chapter section you learned about a corpus, tokens and
document-feature-matrix. Looking at the Stat of the Union corpus,
Britain's manifesto and US Inaugural speeches for document features.
Quanteda has the keyword in context search, and the ability to make
compound tokens.

In the TidyText chapter you learned the tokens, cleaning tokens and to
make word counts. After tokenization and stopwords, sentiment analysis
was done. Using tokens for document term frequencies, n-grams and word
graphs. The section ended with a topic model.

Both chapters follow the same steps:

\begin{itemize}
\tightlist
\item
  get text
\item
  tidy or format text
\item
  tokenize text
\item
  join the stopwords
\item
  join words with sentiment lexicon ``bing'' etc
\end{itemize}

word count on tokenized words, then word count with sentiment word and
sentiment value counts based on words.

\bookmarksetup{startatroot}

\hypertarget{references}{%
\chapter*{References}\label{references}}
\addcontentsline{toc}{chapter}{References}

\hypertarget{refs}{}
\begin{CSLReferences}{0}{0}
\end{CSLReferences}

Text Mining with R: A Tidy Approach. \emph{Julia Silge \& David
Robinson}. \href{https://www.tidytextmining.com/index.html}{Online link}

Quanteda. Quantitative Analysis of Textual Data. \emph{Tutorials}.
\href{https://tutorials.quanteda.io/introduction/}{Online link}



\end{document}
